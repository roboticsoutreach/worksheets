\section{m3pi API Reference}\label{m3pi-api-reference}

There are a lot of functions available in the m3pi library, below is a
reference list for the most useful functions available in the
\href{https://os.mbed.com/users/chris/code/m3pi/docs/4b7d6ea9b35b/classm3pi.html}{m3pi
library}. All of these are functions of your \lstinline!m3pi! instance,
so you will want to call them using \lstinline!m3pi.function()!. For
example, if I want to set the left motor to half speed, I would call
\lstinline!m3pi.left_motor(0.5)!.

\begin{longtable}[]{@{}ll@{}}
\toprule
\begin{minipage}[b]{0.30\columnwidth}\raggedright\strut
Function\strut
\end{minipage} & \begin{minipage}[b]{0.65\columnwidth}\raggedright\strut
Description\strut
\end{minipage}\tabularnewline
\midrule
\endhead
\begin{minipage}[t]{0.30\columnwidth}\raggedright\strut
\lstinline!reset()!\strut
\end{minipage} & \begin{minipage}[t]{0.65\columnwidth}\raggedright\strut
Force a hardware reset.\strut
\end{minipage}\tabularnewline
\begin{minipage}[t]{0.30\columnwidth}\raggedright\strut
\lstinline!left_motor(float speed)!\strut
\end{minipage} & \begin{minipage}[t]{0.65\columnwidth}\raggedright\strut
Directly control the speed and direction of the left motor, accepts
values between -1.0 and 1.0.\strut
\end{minipage}\tabularnewline
\begin{minipage}[t]{0.30\columnwidth}\raggedright\strut
\lstinline!right_motor(float speed)!\strut
\end{minipage} & \begin{minipage}[t]{0.65\columnwidth}\raggedright\strut
Directly control the speed and direction of the right motor, accepts
values between -1.0 and 1.0.\strut
\end{minipage}\tabularnewline
\begin{minipage}[t]{0.30\columnwidth}\raggedright\strut
\lstinline!forward(float speed)!\strut
\end{minipage} & \begin{minipage}[t]{0.65\columnwidth}\raggedright\strut
Drive both motors forwards as the same speed.\strut
\end{minipage}\tabularnewline
\begin{minipage}[t]{0.30\columnwidth}\raggedright\strut
\lstinline!backward(float speed)!\strut
\end{minipage} & \begin{minipage}[t]{0.65\columnwidth}\raggedright\strut
Drive both motors backwards as the same speed.\strut
\end{minipage}\tabularnewline
\begin{minipage}[t]{0.30\columnwidth}\raggedright\strut
\lstinline!left(float speed)!\strut
\end{minipage} & \begin{minipage}[t]{0.65\columnwidth}\raggedright\strut
Drive left motor backwards and right motor forwards at the same speed to
turn on the spot.\strut
\end{minipage}\tabularnewline
\begin{minipage}[t]{0.30\columnwidth}\raggedright\strut
\lstinline!right(float speed)!\strut
\end{minipage} & \begin{minipage}[t]{0.65\columnwidth}\raggedright\strut
Drive left motor forwards and right motor backwards at the same speed to
turn on the spot.\strut
\end{minipage}\tabularnewline
\begin{minipage}[t]{0.30\columnwidth}\raggedright\strut
\lstinline!stop()!\strut
\end{minipage} & \begin{minipage}[t]{0.65\columnwidth}\raggedright\strut
Stop both motors.\strut
\end{minipage}\tabularnewline
\begin{minipage}[t]{0.30\columnwidth}\raggedright\strut
\lstinline!line_position()!\strut
\end{minipage} & \begin{minipage}[t]{0.65\columnwidth}\raggedright\strut
Read the position of the detected line. Returns the position as a float
between -1.0 and 1.0. -1.0 means line is on the left, or the line has
been lost 0.0 means the line is in the middle 1.0 means the line is on
the right.\strut
\end{minipage}\tabularnewline
\begin{minipage}[t]{0.30\columnwidth}\raggedright\strut
\lstinline!sensor_auto_calibrate()!\strut
\end{minipage} & \begin{minipage}[t]{0.65\columnwidth}\raggedright\strut
Calibrate the sensors.\strut
\end{minipage}\tabularnewline
\begin{minipage}[t]{0.30\columnwidth}\raggedright\strut
\lstinline!locate(int x, int y)!\strut
\end{minipage} & \begin{minipage}[t]{0.65\columnwidth}\raggedright\strut
Move the cursor on the screen to (x,y)\strut
\end{minipage}\tabularnewline
\begin{minipage}[t]{0.30\columnwidth}\raggedright\strut
\lstinline!printf()!\strut
\end{minipage} & \begin{minipage}[t]{0.65\columnwidth}\raggedright\strut
Write text to the screen\strut
\end{minipage}\tabularnewline
\bottomrule
\end{longtable}

Note: where a function takes a float, unless otherwise specified, it
takes an input between 0.0 and 1.0 where 0.0 is zero speed (no movement)
and 1.0 is full speed.
