\section{An introduction to m3pi C++
programming}\label{an-introduction-to-m3pi-c-programming}

The \emph{m3pi} bots are programmed in C++, a programming language that
works in a similar way to other languages you may be familiar with such
as Python or Java.

C++ is used mainly for low-level programming where the programmer wants
very specific control of the state of their device, for example when
programming small microcontrollers for watches, motor controllers, or
even robots! Due to the differences between these devices and the more
powerful computers we're used to, some things in C++ work a little bit
differently than what you may be used to:

\subsection{Semicolons}\label{semicolons}

Statements in C++ \textbf{always} end in semicolons, so if I'm just
trying to call a function called \lstinline!doRobotStuff!, I might write
this one line in Python:

\begin{lstlisting}[language=Python]
doRobotStuff()
\end{lstlisting}

In C++, we do the same thing, but we need to make sure there's a
semicolon on the end:

\begin{lstlisting}[language={C++}]
doRobotStuff();
\end{lstlisting}

\subsection{Squiggly Brackets}\label{squiggly-brackets}

In Python, we use colons and indentation to control the flow of our
code. For example, we can write:

\begin{lstlisting}[language=Python]
if condition:
    doAFunction()
    doAnotherFunction()
\end{lstlisting}

In C++, instead of colons and indents, we use brackets, so the
equivalent statement becomes:

\begin{lstlisting}[language={C++}]
if (condition) {
    doAFunction();
    doAnotherFunction();
}
\end{lstlisting}

\subsection{Comments}\label{comments}

Comments are not executed by the code, they are just for you to
understand what is going on.

In Python you can make a comment with

\begin{lstlisting}[language=Python]
# this is a comment which doesn't get executed
\end{lstlisting}

In C++ it's very similar, but you must start with \lstinline!//! instead
of \lstinline!#!.

\begin{lstlisting}[language={C++}]
// this is a single line comment
\end{lstlisting}

You can also create multiple lines of comments with \lstinline!/*! and
\lstinline!*/!

\begin{lstlisting}[language={C++}]
 /* this is a
  multi-line comment */
\end{lstlisting}

\subsection{Data Types}\label{data-types}

In Python we can really easily define variables - we can say
\lstinline!counter = 7! and the variable \lstinline!counter! is set to
the integer value of 7. Python is smart, and knows that when we set a
variable to 7, we want it to be an integer, since 7 is an integer. If I
had said \lstinline!counter = 'seven'! then the \lstinline!counter!
variable would instead be a string.

C++ does things a little differently - it needs to be told what the type
of variable you're declaring. Saying \lstinline!counter = 7! doesn't
work in C++, because it doesn't know what \lstinline!counter! is
supposed to be - it's a variable, but \emph{what kind of variable?} So,
in C++ we need to write \lstinline!int counter = 7;! to let the compiler
know we want an \lstinline!int!, representing the integer data type.

\newpage

C++ data types you may find useful are:

\begin{itemize}
\tightlist
\item
  \lstinline!bool!: either \lstinline!true! or \lstinline!false!.
\item
  \lstinline!int!: a whole number (can be positive or negative).
\item
  \lstinline!float!: a decimal value (does not need to be a whole
  number).
\item
  \lstinline!char!: a single character of text.
\item
  \lstinline!string!: a piece of text consisting of zero or more
  characters.
\end{itemize}

For the \emph{m3pi} bots you'll mostly be using the \lstinline!float!
data type, as most of the values for the bot will be decimal values, and
the \lstinline!float! datatype is the best for representing those.

\subsection{Importing}\label{importing}

In Python we can use other libraries in our code by using the
\lstinline!import! keyword. We can do a similar thing in C++ using
\lstinline!#include!. You probably won't need to do much importing, but
in order to make use of the robot, your program needs to start with
this:

\begin{lstlisting}[language={C++}]
#include "mbed.h"
#include "m3pi.h"
\end{lstlisting}

\subsection{Defining functions}\label{defining-functions}

In Python we define functions using the \lstinline!def! keyword. Here is
a function called \lstinline!addNumbers! that takes two values,
\lstinline!a! and \lstinline!b!, and returns their sum:

\begin{lstlisting}
def addNumbers(a, b):
    result = a + b
    return result
\end{lstlisting}

Here's the equivalent function in C++

\begin{lstlisting}
int addNumbers(int a, int b){
    int result = a + b;
    return result;
}
\end{lstlisting}

The main differences are in the syntax (lots of squiggly brackets and
semicolons), and in the data types; in C++ the function must have a data
type, which is what it will return. Every function must have a data
type, but sometimes we want functions that don't return anything, and
for those we can use the \lstinline!void! data type which lets the
compiler know we're not returning anything. We also need to specify the
data types of the function's parameters, since we're effectively
declaring variables with them.

\subsection{The main function}\label{the-main-function}

In Python, we just make a \lstinline!.py! file and run it, which causes
the code in the file to be run line by line. In C++, the code we want to
start with needs to go in a \lstinline!main! method. So if we write this
in Python:

\begin{lstlisting}[language=Python]
doSomething()
\end{lstlisting}

We can run it immediately. In order to have something we can run
immediately in C++, we would have to write:

\begin{lstlisting}[language={C++}]
int main(){
    doSomething();
}
\end{lstlisting}

Note: the main function returns an int, generally zero if everything
works correctly. You don't need to worry about it for your m3pi robot.

\subsection{Loops}\label{loops}

Both Python and C++ come with loops. The specific loops you're likely to
find useful are \textbf{for loops} and \textbf{while loops}.

\subsubsection{For loops}\label{for-loops}

In Python, we generally use for loops when iterating through a list or a
range of numbers. An example loop in Python, which loops from the
numbers 4, 5, 6, 7, and 8 would be:

\begin{lstlisting}[language=Python]
for counter in range(4, 9):
    doSomething(counter)
\end{lstlisting}

Doing the same in C++ is a tad more complex:

\begin{lstlisting}[language={C++}]
for (int counter = 4; counter < 9; counter ++){
    doSomething(counter);
}
\end{lstlisting}

We see all the usual brackets and semicolons, the bit that's
particularly interesting is
\lstinline!int counter = 4; counter < 9; counter ++!. What's happening
here is that we have 3 separate statements on a single line, the
\emph{initialisation} statement, a \emph{condition} statement, and an
\emph{increment} statement.

The first statement of the loop (the initialisation statement) happens
once, at the start of the loop. In our case, we declare the counter
variable.

The second statement must be true for the loop to continue, in our case
this is checking that the counter is less than 9.

The third statement happens at the end of every loop run, in our case we
just increment the counter by one (this is generally what you want to do
with a for loop).

The simple way to remember this is
\lstinline!for(int variable=START; variable<END; variable++)! (Note:
this is in the Python style where the loop stops at 1 less than END, if
you want it to go all the way, replace the \lstinline!<! condition with
a \lstinline!<=!).

\subsubsection{While loops}\label{while-loops}

While loops are a lot simpler than for loops - the only difference is
syntax. Where in Python we would write:

\begin{lstlisting}[language=Python]
while condition:
    doSomething()
\end{lstlisting}

In C++ we would write:

\begin{lstlisting}[language={C++}]
while (condition){
    doSomething();
}
\end{lstlisting}

\subsection{Other things to consider}\label{other-things-to-consider}

\begin{itemize}
\tightlist
\item
  The common libraries you may be used to in Python like the
  \lstinline!math! library for doing maths do exist to some extent in
  C++, but they will not be identical - you most likely won't need to
  use them though.
\item
  Don't be surprised if the compiler complains the first time round you
  try to compile code. Look through the error message to see what's gone
  wrong, and if you get stuck don't be afraid to ask for help.
\item
  Feel free to experiment - the best way to find out if a language
  behaves a certain way is just to try it.
\end{itemize}
