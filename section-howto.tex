\section{How-To Guides}\label{how-to-guides}

\subsection{Putting code on the robot}\label{putting-code-on-the-robot}

To put code on the robot, simply click `compile' in the compiler. After
a few seconds it should ask where to save the compiled file. Simply plug
the mbed into your laptop and it should show up as a USB stick, choose
to save into the USB stick.

\subsection{Running the robot}\label{running-the-robot}

Unplug the robot, all the lights should turn off. To run the program
just turn on the robot! press the button on the top next to the screen,
and it should start running the code. (You can also press the small
circular button on the mbed to restart the program)

\subsection{How To Guide: Moving}\label{how-to-guide-moving}

The following line can be used to set the robot moving forward.

\begin{lstlisting}
m3pi.forward(speed);
\end{lstlisting}

Replace \lstinline!speed! with a number between 0.0 and 1.0. Motors will
not change until you tell them to! To stop, you should use
\lstinline!m3pi.forward(0)!.

\subsubsection{Setting individual motor
speeds}\label{setting-individual-motor-speeds}

\begin{lstlisting}
m3pi.left_motor(speed);
\end{lstlisting}

and

\begin{lstlisting}
m3pi.right_motor(speed);
\end{lstlisting}

can be used to set the speeds of the two motors separately. You can make
the robot rotate by setting the motors to different speeds

\subsubsection{Rotate left or right on the
spot}\label{rotate-left-or-right-on-the-spot}

Use one of these two lines to rotate on the spot either left or right:

\begin{lstlisting}
m3pi.left(speed);
m3pi.right(speed);
\end{lstlisting}

Replace \lstinline!speed! with a number between 0.0 and 1.0.

\subsection{How to Guide: using the line
sensors}\label{how-to-guide-using-the-line-sensors}

At the start of your program, add:

\begin{lstlisting}
m3pi.sensor_auto_calibrate();
\end{lstlisting}

This will make the robot wiggle to calibrate its light sensor (it must
be on a line!). The calibration needs to be done once every time your
robot is turned on.

You can get the position of the line with the following function:

\begin{lstlisting}
float position_of_line = m3pi.line_position();
\end{lstlisting}

\lstinline!position_of_line! will now contain a decimal (float) value in
the range -1.0 to 1.0 inclusive. It could be any value between -1 and 1!
-1.0 means the line is on the left or cannot be found, 1.0 means the
line is on the right.

\subsection{How to Guide: Using loops}\label{how-to-guide-using-loops}

There are two types of loops, while loops and for loops. Both are types
are shown below, and both print the numbers 0 to 9.

\begin{lstlisting}
int i = 0;
while (i < 10) {
    m3pi.printf("%d", i); // print i
    i++; // increment i by 1
}


for (int j = 0; j < 10; j++) {
    m3pi.printf("%d", j); // print j
}
\end{lstlisting}

You can also loop forever with \lstinline!while (true)!

\subsection{How to Guide: use the LCD
screen}\label{how-to-guide-use-the-lcd-screen}

The LCD screen is a 2x8 character display. To display text, you must
first move the cursor to a point on the screen, and then use
\lstinline!printf! to add to the display.

\begin{lstlisting}
m3pi.locate(0, 1); // move the cursor to the start of the second line (the arguments are (x, y))
m3pi.printf("bees"); // print "bees" on the display
\end{lstlisting}

